\documentclass{beamer}
\usepackage{verbatim}
\usepackage{wrapfig}
\usepackage{color}
%\usetheme{Warsaw}
\usetheme[pageofpages=of,% String used between the current page and the
                         % total page count.
          bullet=circle,% Use circles instead of squares for bullets.
          titleline=true,% Show a line below the frame title.
          %titlepagelogo=opensuse,
          alternativetitlepage=true,% Use the fancy title page.
          ]{Torino}

\title{OS-autoinst: Testing with Perl and openCV}
%\subtitle{Getting the Enlightenment experience \& Being a part of the effort}
\author{\texorpdfstring{Alex-P. Natsios\newline\url{drakevr@2f30.org}}{Author}}
%\author{Alex-P. Natsios}
\institute{YAPC::EU 2014 - sofia}
\date{22 Aug 2014}
\begin{document}
    \begin{frame}
       \titlepage
    \end{frame}

    \begin{frame}{openQA - OS-autoinst}
        \begin{itemize}
            \item openQA is a framework that automatically tests operating systems
            \item It generates keystrokes (like a normal user would)
            \item Fires up Virtual Machine Images via QEMU
            \item It captures images and acts on their contents
            \item It uses libopenCV for fuzzy image matching
            \item It is mostly Perl.
            \item So are the tests.
            \item But the rules are json.
        \end{itemize}
    \end{frame}

    \begin{frame}{050-xterm.pm}
        \tiny
        \verbatiminput{files/xterm.pm}
    \end{frame}

    \begin{frame}{Links}
        \begin{itemize}
            \item https://openqa.opensuse.org/ [Homepage]
            \item http://en.opensuse.org/openSUSE:OpenQA [Wiki Portal]
            \item https://github.com/os-autoinst/os-autoinst [Framework Source]
            \item https://github.com/os-autoinst/openQA [Web Interface]
        \end{itemize}
    \end{frame}

\end{document}
