\documentclass{beamer}
\usepackage{verbatim}
\usepackage{wrapfig}
\usepackage{color}
\usetheme[pageofpages=of,% String used between the current page and the
                         % total page count.
          bullet=circle,% Use circles instead of squares for bullets.
          titleline=true,% Show a line below the frame title.
          alternativetitlepage=true,% Use the fancy title page.
          ]{Torino}

\title{Quick and painless intro to the Perl programming language}
\author{\texorpdfstring{Alex-P. Natsios\newline\url{drakevr@2f30.org}}{Author}}
\institute{GreekLUG}
\date{07 Jun 2015}
\begin{document}
    \begin{frame}
       \titlepage
    \end{frame}

    \begin{frame}
        \center\huge The Basics
    \end{frame}

    \begin{frame}{Introduction}
        \begin{itemize}
            \item What is Perl? 
            \item What is perl? 
            \item What is PERL?
            \item What is Perl's history?
            \item What does it look like?
            \item The motto
            \item Why should I use Perl?
            \item Why shouldn't I use Perl?
        \end{itemize}
    \end{frame}

    \begin{frame}{What is Perl?}
        \begin{itemize}
            \item A high level modern programming language
            \item Open Source and Free
            \item General purpose
            \item Interpreted
            \item Dynamic
        \end{itemize}
    \end{frame}

    \begin{frame}{What is perl?}
        \begin{itemize}
            \item The language Compiler/Interpreter
            \item Compiles and Interprets code on the fly
            \item can be used for simple oneliners
        \end{itemize}
    \end{frame}

    \begin{frame}{What is PERL?}
        \center\huge {\color{red}WRONG!} This is not a canonical or accepted spelling/case and should never be used!
    \end{frame}

    \begin{frame}{What is Perl's history}
        \begin{itemize}
            \item First version released in 1987
            \item Current versions include both Perl 5 and Perl 6
            \item First 5.x version released in 1994
            \item First version for Perl 6 is expected to arrive this year
        \end{itemize}
    \end{frame}

    \begin{frame}{What does it look like?}
        \verbatiminput{files/obfuscated.pl}
    \end{frame}

    \begin{frame}{What does it look like?}
        \begin{itemize}
            \item kidding, that was obfuscated
            \item written like that for compactness, obscurity or just fun.
            \item normal Perl is not written like that.
        \end{itemize}
    \end{frame}

    \begin{frame}{What does it look like?}
        \small
        \verbatiminput{files/lingua_romana_perligata.pl}
    \end{frame}

    \begin{frame}{What does it look like?}
        \begin{itemize}
            \item kidding, that was just a module
            \item written like that for pure fun.
            \item has a paper to accompany and explain it though!
            \item normal Perl is not written like that.
        \end{itemize}
    \end{frame}

    \begin{frame}{What does it look like?}
       \tiny
       \verbatiminput{files/camel.pl}
    \end{frame}

    \begin{frame}{Ok, What does it REALLY look like?}
       \tiny
       \verbatiminput{files/reading_files.pl}
    \end{frame}

    \begin{frame}{The motto}
    \center\huge {\bf There's more than one way to do it}\\
    (TMTOWTDI or TIMTOWTDI, pronounced Tim Toady)
    \end{frame}

    \begin{frame}{Why should I use Perl?}
        \begin{itemize}
            \item General Purpose
            \item Fast and simple to deploy
            \item Fun
            \item Speed is not a critical issue.
            \item Programming freedom (TIMTOWTDI), can code however you like
        \end{itemize}
    \end{frame}

    \begin{frame}{Why shouldn't I use Perl?}
        \begin{itemize}
            \item You have legacy code that is NOT in Perl
            \item Speed is a critical issue
            \item Need FULL optimizable performance
            \item Target system does not have Perl (or cannot install it)
        \end{itemize}
    \end{frame}

    \begin{frame}
        \center\huge Package and Versions Management
    \end{frame}

    \begin{frame}{CPAN}
        \begin{block}{Comprehensive Perl Archive Network}
            \begin{itemize}
                \item Interface for installing modules
                \item Vast collection of Modules and Documentation
                \item Found at \url{http://www.cpan.org}
            \end{itemize}
        \end{block}
    \end{frame}

    \begin{frame}{CPANMINUS}
        \begin{block}{cpanm}
            \begin{itemize}
                \item A self bootstrappable script
                \item Has NO dependencies
                \item Can Get, Unpack, Install CPAN modules
                \item Needs only 10M of RAM when running
                \item Zeroconf, Just works
            \end{itemize}
        \end{block}
    \end{frame}

    \begin{frame}{Perl Binary Management}
        \begin{block}{plenv}
            \begin{itemize}
                \item Simple, UNIX like version management
                \item Needs git and your regular Build tool collection \\
                    like build-essential on debian or the C/C++ development pattern on openSUSE
                \item excellent buddy to Carton (Perl world's ``Bundler'') for dependency management
            \end{itemize}
        \end{block}
    \end{frame}

    \begin{frame}{The community}
        \center\huge
        ``One of Perl's biggest strengths is its community\ldots''
    \end{frame}

    \begin{frame}{Useful links}
        \url{https://www.perl.org/community.html} \\
        \url{http://www.pm.org/} \\
        \url{http://www.perlmonks.com/}
    \end{frame}

    \begin{frame}{Thank you for your attention}
        \center\huge Q \& A
    \end{frame}
\end{document}
